\begin{frame}
    \begin{block}{Anforderungen an Hashfunktionen}
        \begin{itemize}
            \item Hashfunktion berechnet Zahlen aus beliebigen Elementen.
            \item Kollisionen sind nicht zu vermeiden,
                  sollten aber so selten wie möglich auftreten.
            \item Berechnung des Hashes muss schnell gehen.
        \end{itemize}
    \end{block}
    \begin{block}<2->{Vermeidung von Kollisionen}
        \begin{itemize}
            \item Werte sollten möglichst gleichmäßig verteilt sein.
            \item Viel mehr Speicher verwenden als benötigt wird,
                  damit die Wahrscheinlichkeit einer Kollisiton gering ist. 
        \end{itemize}
    \end{block}
    \begin{block}<3->{Umgang mit Kollisionen}
        \begin{itemize}
            \item \alert{geschlossenes Hashing}: Neue Position berechnen (\emph{Sondieren})
            \item \alert{offenes Hashing} Mehrere Elemente pro Position erlauben
            \begin{itemize}
                \item Z.B. als Liste pro Position
            \end{itemize}
        \end{itemize}
    \end{block}
\end{frame}

\begin{frame}
    \begin{block}{geschlossenes Hashing}
        \begin{itemize}
            \item Berechne so lange neue Hash-Werte, bis eine freie Stelle
                  in der Hashtabelle gefunden wurde.
            \item Auch beim Suchen nach Schlüsseln müssen wiederholt Hashes berechnet
                  und die gefundenen Elemente geprüft werden.
        \end{itemize}
    \end{block}
    \begin{block}<2->{Beispiel: Doppel-Hashing}
        \begin{itemize}
            \item Weiche bei Kollisionen auf eine zweite Hashfunktion aus.
            \item Multipliziere die Hash-Werte mit der Anzahl der Versuche.
        \end{itemize}
    \end{block}
    \begin{block}<3->{Beispiel: Kuckucks-Hashing}
        \begin{itemize}
            \item Verwende zwei Hashtabellen mit zwei Hashfunktionen.
            \item Verdränge bei Kollisionen ggf. das vorgefundene Element
            \item Füge verdrängte Elemente in die andere Hashtabelle ein.
        \end{itemize}
    \end{block}
\end{frame}

\begin{frame}
\begin{block}{Zusammenfassung}
    \begin{itemize}
        \item Speichere Elemente in einer Hashmap, bei der Positionen berechnet werden.
        \item Verwende Hashfunktion mit möglichst wenigen Kollisionen.
        \item Verwende viel Speicher, um Kollisionen zu vermeiden.
        \item Bei Kollisionen verwende finde eine neue Position oder speichere Elemente in Listen.
    \end{itemize}
\end{block}
\begin{block}<2->{Eigenschaften von Hashmaps}
    \begin{itemize}
        \item \alert{Effizienz}: Hashfunktion berechnet schnell Positionen.
        \item Average Case: \alert{O(1)} für Suchen und Einfügen.
        \item Worst Case: \alert{O(n)} für Suchen und Einfügen.
        \begin{itemize}
            \item geschlossenes Hashing: Ggf. viele Berechnungen notwendig.
            \item offenes Hashing: Ggf. lange Listen an wenigen Positionen.
        \end{itemize}
    \end{itemize}
\end{block}
    

\end{frame}